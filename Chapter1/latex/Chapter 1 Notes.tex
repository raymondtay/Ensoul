\documentclass[a4paper,12pt]{article}
% PREAMBLE BEGIN
% \usepackage{color}
% PREAMBLE END
\begin{document} %BEGIN
\title{Chapter 1 Notes}
\author{Raymond Tay}
\date{\today}
\maketitle
\tableofcontents
\listoffigures
\listoftables
\newpage
\pagenumbering{arabic}

\section{Introduction}

The heart of any domain model is the set of behaviors or interactions between the various domain elements. These behaviors are at  a higher level of granularity than individual entities or value objects.

\vspace{3mm}

\begin{table}[h!]
\centering
\begin{tabular}{|l|p{0.7\textwidth}|}
\hline
\textsc{Element} & \textsc{Characteristics} \\
\hline
Entity & \begin{itemize}
              \item Has an entity
              \item Passes through multiple states in the lifecycle
              \item Usually has a definite lifecycle in the business
              \end{itemize} \\
\hline
Value Object & \begin{itemize}
\item Semantically immutable
\item Can be freely shared across entities
\end{itemize} \\
\hline
Service & \begin{itemize}
\item More macro-level abstraction than entity or value object
\item Involves multiple entities and value objects
\item Usually models a use case of the business
\end{itemize} \\
\hline
\end{tabular}
\caption{Domain Elements}
\end{table}

\subsection{Domain Element Semantics and Bounded Context}
Let's conclude this discussion on the various domain elements with an important concept that relates their semantics to the bounded context. When we say that an address is a \em{value object}\em, it's a value object only within the scope of the bounded context in which it's being defined.

\subsection{Lifecycle of a domain object}
Every object (entity or value object) that you have in any model must have a definite lifecycle pattern. For every type of object you have in your model, you must have defined ways to handle each of the following events:
\begin{enumerate}
\item \textsc{Creation} - How the object is created within the systme.
\item \textsc{Participation in behaviors} - How the object is represented in memory when it interacts within the system.
\item \textsc{Persistence} - How the object is maintained in the persistent form.
\end{enumerate}

\end{document}   %END
